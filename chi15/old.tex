
  % brief background
  The collection of personal lifestyle data has become ubiquitous in modern society - activity can be tracked using GPS, sleep patterns can be recorded using wristbands and a deluge of mobile apps exist to track dieting. The collection of such detailed personal data is known as personal informatics, a field that is becoming increasingly pertinent in Healthcare. Personal informatics have been demonstrated as promoting positive health behaviours within individuals, by encouraging people to walk more [], eat less [] and live more healthily []. However, current research is limited to a \textit{collection-reflection} model, in which individuals act upon their own data. This dissertation builds on this model, instead considering a \textit{collection-observation} model, in which healthcare practitioners may use an individual's data to support decision making.


  The use of patient data collection forms the foundation of evidence-based healthcare (WRONG!), which has been considered increasingly important as a way to encourage rational decision making \citep{Heneghan2013}, leading to less mistakes, and in turn reducing mortality. However, its use has been limited to bespoke data recording techniques which puts a significant burden on the patient to record their own activities. Such limitations have led it to be described as an ideal, but unrealistic \citep{Heneghan2013}. However, this dissertation argues that, with modern devices, such data recording has become more ubiquitous, making evidence based healthcare more accessible and reasonable. Cognitive biases identified in diagnoses - inportant to understand and prevent \citep{Croskerry2003}. Furthermore, its pervasiveness may enable more widespread adoption, in turn reducing mortality within healthcare scenarios. Diagnostic error is associated with morbity - it is preventable \citep{Graber2002}.  has called for efforts to prevent diagnostic mistakes, though understanding and minimisation of cognitive errors is the greatist challenge.


  % methodology
  Due to the multidisplinary nature of technology and healthcare, a Web Science approach has been applied to understanding bias within this space. As such, this dissertation will take the form of a literature review, considering material from both health sciences, computer science and cognitive sciences to build a picture of the dangers of bias in personal informatics within healthcare. From this, 16 biases have been identified, and their risks considered in two forms of healthcare scenarios -- those that are resource-rich, and those which are thin-slices. This dissertation proposes that biases more prevalent within resource-rich scenarios pose a large risk to decision-making, and that risks may be reduced through the design of personal informatics and careful consideration into how personal informatics may be implemented within Healthcare. Draws from cognitive dispositions to respond (biases which affect diagnostics) and biases which affect how data is interpreted.


  This dissertation make three key research contributions. The first is two health sciences, in considering how personal informatics might be applied to healthcare scenarios and the risks of bias within these. The section is to technology, in identifying areas of design that lead on to increased risks of biases. Finally, the third is to Web Science, in proposing a literature synthesis methodology to understanding the greater unseen complexities of using technology within healthcare.

  % Future work
  In understanding the risks of biases, this dissertation proposes a study, which will investigate the prevalence and significance of bias in using personal informatics within healthcare. A design is put forward and critiqued, with a descussion of the limitations and ethical issues. It is hoped that this study will be conducted with nurses and stroke patient, as their rehabilitation involves the observations of everyday lifestyle activities.


The use of patient data collection forms the foundation of evidence-based healthcare (WRONG!), which has been considered increasingly important as a way to encourage rational decision making \citep{Heneghan2013}, leading to less mistakes, and in turn reducing mortality. However, its use has been limited to bespoke data recording techniques which puts a significant burden on the patient to record their own activities. Such limitations have led it to be described as an ideal, but unrealistic \citep{Heneghan2013}. However, this dissertation argues that, with modern devices, such data recording has become more ubiuqitous, making evidence based healthcare more accessible and reasonable. Furthermore, its pervasiveness may enable more widespread adoption, in turn reducing mortality within healthcare scenarios.
      % Thinking Fast and Slow
      % systematic error - bias
      % confidence speaker - audience more easily impressed
      % judgement/decision making
      % stastistics ignored
      % predictable heuristics
      % farmer or librarian? more farmers, but librarian nearly always chosen
      % biases of intuitive thinking
      % availability heuristics
      % violate rules of rationale choice
      % decision making under uncertainty
      % data may prevent this?
      % systematic errors
      % prolonged practice --> accurate intuition of experts
      % emotion - affect heuristic
      % prospect theory - decision under risk
      % behavioral economics



  This dissertation make three key research contributions. The first is two health sciences, in considering how personal informatics might be applied to healthcare scenarios and the risks of bias within these. The section is to technology, in identifying areas of design that lead on to increased risks of biases. Finally, the third is to Web Science, in proposing a literature synthesis methodology to understanding the greater unseen complexities of using technology within healthcare.

 Due to the multidisplinary nature of technology and healthcare, a Web Science approach has been applied to understanding bias within this space. As such, this dissertation will take the form of a literature review, considering material from both health sciences, computer science and cognitive sciences to build a picture of the dangers of bias in personal informatics within healthcare. From this, 16 biases have been identified, and their risks considered in two forms of healthcare scenarios -- those with large context, and those will little context. This dissertation proposes that biases more prevalent within little context scenarios pose a large risk to decision-making, and that risks may be reduced through the design of personal informatics and careful consideration into how personal informatics may be implemented within Healthcare.



      In terms of patient data, biases affect how data is interpreted (which I will call Interpretation Biases). Those biases thus get carried through to clinical decision making. Decisions are therefore affected by both CDRs and Interpretation Biases, as shown in Figure~\label{fig:clinicdata}.



      %\begin{enumerate}
      %\item presenting too much data may overwhelm clinicians, leading to focus on smaller parts of the data (illusory patterns, aggregate bias)
      %\item presenting data in such a way that only trained experts can understand it can lead to incorrect observations (ambiguity effect, outcome bias, confirmation bias)
      %\item ... other recommendations based on high-risk biases
      %\end{enumerate}

      %Resource-rich scenarios, while at a lower risk of most CIRBs, have been identified to still at risk of confirmation bias, framing, and, in particular, functional fixedness.
      %\begin{enumerate}
      %\item more design recommendations
      %\end{enumerate}
      %Limitations

      %Automatic \citep{Kahneman2012} → System 1 \citep{Kahneman2012}
      %Reflective \citep{Kahneman2012} → System 2 \citep{Kahneman2012}

      %Thin-slice

      %Less knowledge \citep{Mussweiler2000}
      %Subjective

      Availability [High]
      immediate examples that come to mind. ``employed when people are asked to assess the frequency of a class or the plausibility of a particular development'' \citep{Kahneman1974}. ``ebola'' Nudge p24. Information which is considered more important, such as recent information, is heavy relied upon for decision making

      Ambiguity [High]
      lack of information (ambiguity). What is the
      best expectation I might associate with this action, without being unreasonable? \citep{Ellsberg1961a}

      Representativeness [High]
      judgements about probability made under uncertainty.
      ``employed when people are asked to judge the probability that an object or event A belongs to class or process B'' \citep{Kahneman1974}
      Nudge p26

      Anchoring [High]
      Focusing effect - placed importance on one aspect of an event. Heavy reliance on first piece of information (anchor). ``employed in numerical prediction when a relevant value is available''  \citep{Kahneman1974}. caused by priming
      Nudge p23

      Clustering [High]
      Tversky, hot hand, betting based on basketball streaks. \citep{Kahneman1974}
      Caused by representativeness heuristic

      confirmation bias [High]
      Favor judgment which confirms one’s beliefs.

      contrast effect [high]
      repeated tasks changes performance.
      related to ego depletion?


      %Greater knowledge \citep{Mussweiler2000}
      %Objective



      Extension neglect [High]
      Base rate neglect
      focus on one anomaly, related to representativeness heuristic

      curse of knowledge [High]
      Better informed people find it difficult to think about problems from the perspective of lesser informed people.

      framing [High]
      Nudge p36
