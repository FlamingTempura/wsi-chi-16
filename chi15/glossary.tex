
BIASES

\item[Bias]: Used to collectively include cognitive biases and heuristics.

\item[Interpretation Biases] (IBs): Biases which affect data interpretation.

\item[Cognitive Dispositions to Respond] (CDRs): Biases which affect clinical decision making

\item[Clinical Interpretation Reasoning Biases] (CIRBs): Biases which affect evidence-based clinical decision making.


ACTORS

\item[Clinician]: The medical worker who is making decisions about a patient. Instances: health worker, physician, practitioner, nurse.

\item[Patient]: The person who is seeing a clinician.


ACTIONS

\item[Clinical Decision]: A decision made by a clinician. Instances: diagnoses.

\item[Clinical Decision Making]: The process of a clinical making a clinical decision. Affected by CDRs.

\item[Interpretation]: Making sense of data - weather in numerical, textual or graphical form. Affected by IBs.

\item[Evidence-based Clinical Decision]: A decision informed by evidence, which we assume to have been interpreted from patient data.

\item[Evidence-based Clinical Decision Making]:

\item[Judgement]: An action which uses a cognitive process. Instances: Interpretation, Clinical Decision.


SCENARIOS

\item[Clinical Scenario]: A scenario in which a decision must be made, with varying availability of certainty, knowledge, time and informational resources.

\item[Resource-Rich Scenario]: A scenario where there is sufficient time, access to supplementary information and clinician-patient history. Instances: GP

\item[Thin-Slice Scenario]: A scenario where there is heavy time constraints, little access to information and no clinical patient history. Instances: Emergency room

\item[Adverse Event]: A scenario in which there are undesirable outcomes which may have been the result of a biased decision.


EVIDENCE

\item[Presenting complaints]: Immediate evidence available to a clinician due to the presence of the patient.

\item[Patient Data]: numerical, textual or graphical information about a patient which is provided externally (i.e., may not necessarily be with the patient at all times). Example instances: medical record, quantified self data, lifelog.

\item[Personal lifestyle data]


OTHER
\item[Emergency Room]
