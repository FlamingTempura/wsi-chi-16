\documentclass{sigchi}

% Use this command to override the default ACM copyright statement
% (e.g. for preprints).  Consult the conference website for the
% camera-ready copyright statement.


%% EXAMPLE BEGIN -- HOW TO OVERRIDE THE DEFAULT COPYRIGHT STRIP -- (July 22, 2013 - Paul Baumann)
% \toappear{Permission to make digital or hard copies of all or part of this work for personal or classroom use is      granted without fee provided that copies are not made or distributed for profit or commercial advantage and that copies bear this notice and the full citation on the first page. Copyrights for components of this work owned by others than ACM must be honored. Abstracting with credit is permitted. To copy otherwise, or republish, to post on servers or to redistribute to lists, requires prior specific permission and/or a fee. Request permissions from permissions@acm.org. \\
% {\emph{CHI'14}}, April 26--May 1, 2014, Toronto, Canada. \\
% Copyright \copyright~2014 ACM ISBN/14/04...\$15.00. \\
% DOI string from ACM form confirmation}
%% EXAMPLE END -- HOW TO OVERRIDE THE DEFAULT COPYRIGHT STRIP -- (July 22, 2013 - Paul Baumann)


% Arabic page numbers for submission.  Remove this line to eliminate
% page numbers for the camera ready copy 

%\pagenumbering{arabic}

% Load basic packages
\usepackage{balance}  % to better equalize the last page
\usepackage{graphics} % for EPS, load graphicx instead 
%\usepackage[T1]{fontenc}
\usepackage{txfonts}
\usepackage{times}    % comment if you want LaTeX's default font
\usepackage[pdftex]{hyperref}
% \usepackage{url}      % llt: nicely formatted URLs
\usepackage{color}
\usepackage{textcomp}
\usepackage{booktabs}
\usepackage{ccicons}
\usepackage{todonotes}
% \usepackage{natbib}

% llt: Define a global style for URLs, rather that the default one
\makeatletter
\def\url@leostyle{%
  \@ifundefined{selectfont}{\def\UrlFont{\sf}}{\def\UrlFont{\small\bf\ttfamily}}}
\makeatother
\urlstyle{leo}

% To make various LaTeX processors do the right thing with page size.
\def\pprw{8.5in}
\def\pprh{11in}
\special{papersize=\pprw,\pprh}
\setlength{\paperwidth}{\pprw}
\setlength{\paperheight}{\pprh}
\setlength{\pdfpagewidth}{\pprw}
\setlength{\pdfpageheight}{\pprh}

% Make sure hyperref comes last of your loaded packages, to give it a
% fighting chance of not being over-written, since its job is to
% redefine many LaTeX commands.
\definecolor{linkColor}{RGB}{6,125,233}
\hypersetup{%
  pdftitle={SIGCHI Conference Proceedings Format},
  pdfauthor={LaTeX},
  pdfkeywords={SIGCHI, proceedings, archival format},
  bookmarksnumbered,
  pdfstartview={FitH},
  colorlinks,
  citecolor=black,
  filecolor=black,
  linkcolor=black,
  urlcolor=linkColor,
  breaklinks=true,
}

% create a shortcut to typeset table headings
% \newcommand\tabhead[1]{\small\textbf{#1}}

% End of preamble. Here it comes the document.
\begin{document}

\title{The Quantified Patient in the Doctor's Office: Challenges in Designing for Clinical Decision-Making Settings}

\numberofauthors{3}
\author{%
  \alignauthor{1st Author Name\\
    \affaddr{Affiliation}\\
    \affaddr{City, Country}\\
    \email{e-mail address}}\\
  \alignauthor{2nd Author Name\\
    \affaddr{Affiliation}\\
    \affaddr{City, Country}\\
    \email{e-mail address}}\\
  \alignauthor{3rd Author Name\\
    \affaddr{Affiliation}\\
    \affaddr{City, Country}\\
    \email{e-mail address}}\\
}

\maketitle

\begin{abstract}
While the \emph{Quantified Self} movement has largely centred around the individual’s own use of his or her data, the same kinds of patient-captured information about their well-being could, in theory, facilitate more accurate and effective medical diagnosis and treatment.  In practice, however, introducing such data during in a way that they can be used effectively and without additional risk during patient visits and hospital episodes creates significant challenges, that we believe, HCI research can help.  In this paper, we seek to understand the primary bottlenecks to the use of QS data during patient visits in both primary and secondar (specialist) care through a literature survey and group 

\end{abstract}

\keywords{Quantified Self; mHealth; Clinical Decision-Making}

\category{H.5.m.}{Information Interfaces and Presentation
  (e.g. HCI)}{Miscellaneous} \category{See
  \url{http://acm.org/about/class/1998/} for the full list of ACM
  classifiers. This section is required.}{}{}

\section{Introduction}

Empowering patients ``take charge'' of their health is an idea frequently championed by politicians, technologists, and healthcare experts alike~\cite{swan2012health}.  Yet, despite countless government and industry-led initiatives across Europe, the UK, and North America, to inspire this ``patient-led healthcare revolution'', it has yet to happen.  

One area, however, where individuals have been taking the lead in trying to understand their own health, is the \emph{Quantified Self} movement \cite{}, primarily comprised of hobbyists and non-health experts who use technological tools to meticulously record and interrogate the minutiae of their physical and mental states over time.   As the population of those interested in self-tracking grew and attracted mainstream interest, the industry has swiftly responded to the expanding demand for self-tracking tools and technologies with a now enormous collection of wearable and embeddable sensors.  These technologies now enable people to record and keep track of aspects of their health with less effort and better fidelity than ever previously, and will continue to improve such as by being less invasive, more comfortable, and more accurate.

While the direct application of such sensors to understanding patients' particular symptoms, situations and lifestyles in the healthcare context would seem straightforward, most physicians today avoid or outright reject the use of self-logged QS data. Even when this information is provided by patients and is reference, it is rarely, if ever used for differential diagnosis.  

What are the barriers to the use of self-logged QS data in critical clinical decision making settings?  What are doctors, nurses, specialists and medical professionals' views on the QS movement and self-logged data?  Is it a problem of what is being captured, how it is being captured, how it is represented, or presented?  In this paper, we summarise an initial exploration of these questions comprised of two stages: first, a literature survey pre-study, which informed the design of a set of role-playing probes with clinical specialists.

\section{Background} 

The proliferation of consumer devices for health tracking have led calls to integrate consumer technology within healthcare. In the UK, for example, the Personalised Health and Care 2020 framework set out a vision in which health and wellbeing data, sensed from wearable and environmental sensors, and mobile health apps, will form part an integral part of patient health records by 2018~\cite{Personalised2014}.  The framework proposes that these data, by ``filling in the gaps'' between visits with their GP will enable them to more effectively perform differential diagnosis at point of admission, thereby simultaneously improving outcomes and driving down costs.  By applying data mining methods to such self-logged data, it will enable early onset detection of chronic conditions, thereby potentially reversing the course of some conditions before they set in~\cite{Swan2009}.  Such data may also lead to medical insights pertaining to the complex health, environmental and lifestyle factors that might contribute to patients' conditions, providing a mechanism for better understanding diseases and their linked causes.  Initial enthusiasm for this vision is also evident in the US, with the US Food and Drug Administration's approval of  consumer tracking devices for clinical trials, citing the importance of quantifiable analysis of physical activity to physiological monitoring~\cite{U.S.FoodandDrugAdministration2014}.

Yet despite this early enthusiasm for use of self-logged QS data in clinical practice, uptake has been slow.  Industry initiatives, for example, Apple's  partnership with several US healthcare providers and hospitals has centred around targeted use in clinical trials rather than in differential diagnosis; here, still, it has met resistance, with some partners citing insufficient interest by both patients and physicians~\cite{}.  But why?

An informal survey by VentureBeat found that clinicians were largely ''simply not interested in FitBit data'', assigning blame on several factors~\cite{venturebeat}. The first was time; doctors cited simply not having the time to look at data brought in by patients.  The second pertained ot access; electronic medical records simply did not admit or integrate patient-supplied data.  Third were concerns about data quality, e.g. that devices available today were simply not yet accurate enough for clinical use.  Finally, doctors expressed legal concerns around use of the data; citing the potential to be held liable for data captured by wearable health sensors.

Since multiple factors seemed to be potentially contributing to the slow uptake of patient-logged data in clinical practice, including temporal, situational, social, and potentially even legal constraints, we started our investigation with a broad survey of medical literature focusing on the use of data during visits with patients and hospital admissions.  


% a little bit about all the potential ways QS could help

% \cite{Swan2009} has called for use of this data as a means to supplement healthcare, which would increase ``information flow, transparency, customization, collaboration and patient choice and responsibility taking, as well as quantitative, predictive and preventive aspects.''. The use of self-tracking data in healthcare is not new. The health diary has been popular since the 1950's as a data collection method \cite{Richardson1994}.  \cite{Richardson1994} note that particular types of data that are recorded include pain, fatigue, medication use and dietary intake. This shares overlap with health and wellbeing data, suggesting there are existing use cases where health and wellbeing data could be supplemented. \cite{OLoughlin2013} has demonstrated the use of lifelogging for the purpose of increasing diet awareness. However, \cite{Swan2009} has discussed that there is currently little adoption, perhaps due to the barriers of technology in clinical environments.


\section{Pre-study: Literature Survey} 

We anticipated that the slow uptake of the use of self-logged QS could potentially be attributed to a large number of factors, and thus decided it prudent to first do a survey of medical, quantified-self and HCI literature.  Our objective  was to establish a theoretical framework from which we could then identify areas that could require further investigation, as well as opportunities where research in HCI could potentially help.  

\subsection{Review Method}

We started with a set of search terms broad enough to encompass studies of clinical practice where patient-logged data (both paper-based and digital, manual and automatic) were introduced into a clinical setting.  To do this we searched PubMed, Google Scholar and the ACM DL for keywords ``patient diaries'', ``care diaries'', ``wellbeing diaries'', ``self-report diaries'', ``quantified-self'', ``smartphone apps'', and ``wearable sensors''.

Since we wanted to focus on the usage of data by medical experts, we excluded studies about use by patients themselves, such as for feedback, reflection, goal setting and self monitoring, including behaviour-change studies and studies of motivation to self-diaries, which were prevalent from the HCI community.   Focusing only on existing practice, we omitted papers describing new interfaces and systems that have not had substantial adoption.  We also excluded papers discussing the capture side of health diarising and life-logging by patients, except where aspects of capture affected its later use.  We were careful to include papers that discussed any issues relating to the use of patient data in clinical setting, including those that discussed human factors issues specifically, to more broadly  theatres and emergency rooms.

After finding a small set of results, we then broadened our search to include studies that discussed the use of patient data in medical decision-making, including both patient-supplied and clinical data held by providers themselves.  We included ``telemonitoring'', and ''electronic patient records''.  This  hoped to find a broad range of factors spanning human-factors issues to social, cultural, institutional, situational, among others.

For each paper, we identified problem(s) in the clinical use of data, which were first added to a spreadsheet and linked to their original source.  After examining each of the papers, 2 researchers organised the list into themes, attempting to merge all problems with the same underlying cause, while keeping those that did not overlap distinct.

\subsection{Results}

From an initial set of 2340 results, we identified 429 papers that actually contained at least one of the search terms among keywords and the abstract. From here, we identified $XX$ relevant papers according to the criteria defined above.  From this we derived the $CC$ themes visible in \ref{fig:themes}.

\subsection{Results: Cognitive Dispositions to Respond}

During the first phase, we encountered a substantial body of work pertaining to  situational and psychological factors that influenced the decision-making performance of medical experts in clinical settings.  Croskerry's \emph{Cognitive Dispositions to Respond}~\cite{} 

% We started with MEDLINE, Web of Science, and the ACM DL
% Using the keywords "medical decision-making", "clinical decision-making", "patient-logged data", "quantified self", "data in hospital admission", "GP visit", searches were performed on BMJ, MEDLINE, CINAHL, Google Scholar, Web of Science, Springer, and the ACM DL.  This collectively resulted in 2982 results, many of which
% d
% Temporal constraints
% Situational constraints
% UI/UX constraints 
% Workflow constraints
% Cognitive Biases
% Training/background constraints
% Exclusion criteria:
%   Anything that only involved the patient, including behaviour change studies


%   UK Prime Minister Gordon Brown,  pronounced in 2008 that more effective ``future will be one of patient power, patients engaged and taking control over their own health and healthcare.” – Gordon Brown, U.K. Prime Minister



% functional fixedness - doctors _did_ use the data for other 
% confirmation bias - using particular data sources to support a particular hypothesis
% framing
% functional fixedness

\section{Methodology}

May - June 2015

Inspired by methodology from other research

A focus group and two interviews were conducted

Recorgnize the value of self-logged data



\subsection{Context and Participants}

Participants in the focus group - eight specialists at Mt Sinai (six male)

Participants in interviews - two NHS GPs

\subsection{Procedure}

Two interviews

One focus group

\subsection{Data Analysis}


\section{Results}

Can we rely on the data?

\subsection{Summary of findings}


\section{Discussion}

\subsection{Implications for design}


\section{Conclusions}


% REFERENCES FORMAT
% References must be the same font size as other body text.
\bibliographystyle{SIGCHI-Reference-Format}
\bibliography{wsi-chi-16}

\end{document}

%%% Local Variables:
%%% mode: latex
%%% TeX-master: t
%%% End:
