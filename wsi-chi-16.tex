\documentclass{sigchi}

% Use this command to override the default ACM copyright statement
% (e.g. for preprints).  Consult the conference website for the
% camera-ready copyright statement.


%% EXAMPLE BEGIN -- HOW TO OVERRIDE THE DEFAULT COPYRIGHT STRIP -- (July 22, 2013 - Paul Baumann)
% \toappear{Permission to make digital or hard copies of all or part of this work for personal or classroom use is      granted without fee provided that copies are not made or distributed for profit or commercial advantage and that copies bear this notice and the full citation on the first page. Copyrights for components of this work owned by others than ACM must be honored. Abstracting with credit is permitted. To copy otherwise, or republish, to post on servers or to redistribute to lists, requires prior specific permission and/or a fee. Request permissions from permissions@acm.org. \\
% {\emph{CHI'14}}, April 26--May 1, 2014, Toronto, Canada. \\
% Copyright \copyright~2014 ACM ISBN/14/04...\$15.00. \\
% DOI string from ACM form confirmation}
%% EXAMPLE END -- HOW TO OVERRIDE THE DEFAULT COPYRIGHT STRIP -- (July 22, 2013 - Paul Baumann)


% Arabic page numbers for submission.  Remove this line to eliminate
% page numbers for the camera ready copy 

%\pagenumbering{arabic}

% Load basic packages
\usepackage[table]{xcolor}
\usepackage{balance}  % to better equalize the last page
\usepackage{graphics} % for EPS, load graphicx instead 
%\usepackage[T1]{fontenc}
\usepackage{txfonts}
\usepackage{times}    % comment if you want LaTeX's default font
\usepackage[pdftex]{hyperref}
% \usepackage{url}      % llt: nicely formatted URLs
\usepackage{color}
\usepackage{textcomp}
\usepackage{booktabs}
\usepackage{ccicons}
\usepackage{todonotes}
% \usepackage{natbib}
\usepackage{array}

% llt: Define a global style for URLs, rather that the default one
\makeatletter
\def\url@leostyle{%
  \@ifundefined{selectfont}{\def\UrlFont{\sf}}{\def\UrlFont{\small\bf\ttfamily}}}
\makeatother
\urlstyle{leo}

% To make various LaTeX processors do the right thing with page size.
\def\pprw{8.5in}
\def\pprh{11in}
\special{papersize=\pprw,\pprh}
\setlength{\paperwidth}{\pprw}
\setlength{\paperheight}{\pprh}
\setlength{\pdfpagewidth}{\pprw}
\setlength{\pdfpageheight}{\pprh}

% Make sure hyperref comes last of your loaded packages, to give it a
% fighting chance of not being over-written, since its job is to
% redefine many LaTeX commands.
\definecolor{linkColor}{RGB}{6,125,233}
\hypersetup{%
  pdftitle={SIGCHI Conference Proceedings Format},
  pdfauthor={LaTeX},
  pdfkeywords={SIGCHI, proceedings, archival format},
  bookmarksnumbered,
  pdfstartview={FitH},
  colorlinks,
  citecolor=black,
  filecolor=black,
  linkcolor=black,
  urlcolor=linkColor,
  breaklinks=true,
}

% create a shortcut to typeset table headings
% \newcommand\tabhead[1]{\small\textbf{#1}}

% End of preamble. Here it comes the document.
\begin{document}

\title{The Quantified Patient in the Doctor's Office: Challenges in Designing for Clinical Decision-Making Settings}

%\numberofauthors{4}
%\author{%
%  \alignauthor{Peter West\\
%    \affaddr{Web Science}\\
%    \affaddr{University of Southampton, UK}\\
%    \email{p.west@soton.ac.uk}}\\
%  \alignauthor{Dr. Max Van Kleek\\
%    \affaddr{Computer Science}\\
%    \affaddr{University of Oxford, UK}\\
%    \email{max.van.kleek@cs.ox.ac.uk}}\\
%  \alignauthor{Professor Sir Nigel Shadbolt\\
%    \affaddr{Computer Science}\\
%    \affaddr{University of Oxford, UK}\\
%    \email{nigel.shadbolt@cs.ox.ac.uk}}\\
%\alignauthor{Dr. Richard Giordano\\
%    \affaddr{Health Science}\\
%    \affaddr{University of Southampton, UK}\\
%    \email{r.giordano@soton.ac.uk}}\\
%}

\numberofauthors{2}
\author{%
  \alignauthor{Peter West, Richard Giordano\\
    \affaddr{Faculty of Health Sciences}\\
    \affaddr{University of Southampton, UK}\\
    \email{\{p.west,r.giordano\}@soton.ac.uk}}\\
  \alignauthor{Max Van Kleek, Nigel Shadbolt\\
    \affaddr{Department of Computer Science}\\
    \affaddr{University of Oxford, UK}\\
    \email{\{max.van.kleek,nigel.shadbolt\}@cs.ox.ac.uk}}\\
}

\maketitle

\begin{abstract}
While the \emph{Quantified Self} movement has largely centred around the individual’s own use of his or her data, the same kinds of patient-captured information about their well-being could, in theory, facilitate more accurate and effective medical diagnosis and treatment.  In practice, however, introducing such data during in a way that they can be used effectively and without additional risk during patient visits and hospital episodes creates significant challenges, that we believe, HCI research can help.  In this paper, we seek to understand the primary bottlenecks to the use of QS data during patient visits in both primary and secondar (specialist) care through a literature survey and group 

\end{abstract}

\keywords{Quantified Self; mHealth; Clinical Decision-Making}

\category{H.5.m.}{Information Interfaces and Presentation
  (e.g. HCI)}{Miscellaneous} \category{See
  \url{http://acm.org/about/class/1998/} for the full list of ACM
  classifiers. This section is required.}{}{}

\section{Introduction}

Empowering patients ``take charge'' of their health is an idea frequently championed by politicians~\cite{brown, obama}, technologists~\cite{ihealth}, journalists~\cite{goetz} and healthcare experts alike~\cite{swan2012health}.  Yet, despite countless government and industry-led initiatives across Europe, the UK, and North America, to inspire this ``patient-led healthcare revolution'', it has yet to happen.  

One area, however, where individuals have been taking the lead in trying to understand their own health, is the \emph{Quantified Self} movement \cite{}, primarily comprised of hobbyists and non-health experts who use technological tools to meticulously record and interrogate the minutiae of their physical and mental states over time.   As the population of those interested in self-tracking grew and attracted mainstream interest, the industry has swiftly responded to the expanding demand for self-tracking tools and technologies with a now enormous collection of wearable and embeddable sensors.  These technologies now enable people to record and keep track of aspects of their health with less effort and better fidelity than ever previously, and will continue to improve such as by being less invasive, more comfortable, and more accurate.

While the direct application of such sensors to understanding patients' particular symptoms, situations and lifestyles in the healthcare context would seem straightforward, most physicians today avoid or outright reject the use of self-logged QS data. Even when this information is provided by patients and is reference, it is still rarely, if ever, used for differential diagnosis.  

What are the barriers to the use of self-logged QS data in critical clinical decision making settings?  This is a delicate question to approach for several reasons; first, during the a course of a single patient visit, there may be many kinds of decisions made by a plurality of medical professionals in different roles, in similarly diverse settings.   Paramedics in an ambulance, triage nurses within an emergency room, specialists in a acute care unit or hospital wards, to general practitioners (GPs) in their surgeries, all make decisions regarding a patient under very distinct situational and informational constraints~\cite{croskerry}. Second, even if focus is centred around a single setting, by a single class of medical professionals, such as GP surgeries, there may be significant disparities in the ways different GPs do things; for example, the degree to which GPs use electronic medical records (EMRs)~\cite{hunt} to organise patient data, prescribe particular tests or treatments~\cite{}, or the mechanism and method that they use to maintain good patient relationships~\cite{} can be highly inconsistent. Finally, clinical decision making is a field that is not only long established, but one of the most specialised occupations in the modern workforce; therefore, understanding the effects of the introduction of any new information into such a setting requires an in-depth understanding of the processes, training and tradition.

For our investigation, we focused on two different roles: primary care physicians on the ``frontline'' of the medical service, and secondary care specialists who work in hospitals in specialised units.  In our investigation, we sought to understand the fundamental nature of the barriers to adoption of broadly-classed 'self-logged' patient data, particularly of the most common types facilitated by the many kinds of consumer health monitoring tools becoming available.  That is, we wished to identify whether reasons that such self-logged data were still, so often, deemed useless by clinical professionals, included \emph{what} was captured, how it was captured, the ways in which it was represented or made available during a patient consultation, or  due to other, yet unidentified problems derived from looking at or processes required to derive insight from it.  From this, we wished to extrapolate how such problems might help be shaping and solved through HCI research, such as by re-thinking tools people used to monitor themselves, the kinds of data they captured, or the ways that physicians and medical professionals might access and use it, towards the visions of eHealth and mHealth set out so often in the popular press (e.g. ~\cite{goetz, thingy, thingy}).

\section{Overview of Approach}

We began with an initial set of broad dimensions (Figure \ref{fig:qs}) chosen to represent the potentially many kinds of reasons that self-logged, QS data remains far from fulfilling its potential in clinical practice.

\begin{figure}[tbpb]
\begin{enumerate}
    \item \emph{Data subject}: Lack of relevance or utility of \emph{what} is captured
    \item \emph{Sampling method}: Problems with \emph{how} and \emph{when} information is captured
    \item \emph{Accuracy}: Lack of trust in accuracy of instruments or capture method
    \item \emph{Access}: Problems with physician \emph{access} to the data
    \item \emph{Patient communication}: Interference with patient clinician communication
    \item \emph{Workflow}: Interference or exclusion from diagnostic workflow
    \item \emph{Biases}: Danger of cognitive biases introduced by data?
    \item \emph{Other}: External factors? (\emph{data handling}, \emph{legal}, \emph{billing} etc)
\end{enumerate}
\caption{Initial dimensions for categorising barriers to clinical use of self-logged/QS data.}
\label{fig:qs}
\end{figure}

Since these dimensions encompassed a broad set of possible factors, including data-oriented problems (such as pertaining to subject, quality and sampling), situational constraints, and practice constraints, we wished to understand which, if any, of these dimensions had support from previous studies.  This lead us a broad survey of medical literature, which we describe in the next section.  

Then, to investigate whether we could gather primary evidence towards the importance of any of these dimensions, we followed with a role-play study in which practicing GPs and specialists were presented with patient illness scripts, based on real cases, in which patients brought various QS data with them while in the process of presenting their complaint.  We asked GPs and specialists to come up with their own diagnosis, following a think-aloud protocol.  We then transcribed, annotated and derived codes from each session to identify reasons the QS data were or were not consulted, and ways that the presence of the QS data influenced the diagnostic process.  From these codes, we derived a set of themes, which we then used to compare our findings to previous findings in the literature.  Finally, we discuss the limitations of our study and future work.

% To undestand this method we conducted a two phase invesitgation. Fi


% This paper presents a collective summary of our initial investigation in 

%Second, medical decision-making is making  we wish to approach this question carefully, considering What are doctors, nurses, specialists and medical professionals' views on the QS movement and self-logged data?  Is it a problem of what is being captured, how it is being captured, how it is represented, or presented?  In this paper, we summarise an initial exploration of these questions comprised of two stages: first, a literature survey pre-study, which informed the design of a set of role-playing probes with clinical specialists.


\section{Background} 

We contextualise our investigation against two closely related fields: the first, \emph{evidence based medicine} sought to expand the use of empirical methods (such as used in epidemiology) for evaluating and improving the effectiveness of decisions such as those affecting treatment choices.  The second, \emph{clinical decision making} examines the cognitive, interactional and situational processes which influence how practitioners arrive at decisions, under the practical constraints necessary for conducting their practices.  In this section, we first introduce at a high level, common visions of how technology-enabled, patient data-driven healthcare might be realised in the future.  We then discuss evidence-based medicine and the role of clinical decision making literature to our investigation.

\subsection{Visions for Data-driven eHealth / mHealth}
Notions of ``big data'' and ``data driven healthcare'' have inspired many visions  for more data-informed healthcare internationally. In the popular press, Thomas Goetz's \emph{The Decision Tree} outlined a vision in which every person will be DNA-tested at birth, and tracked with sensors throughout their lives; the resulting piles of data would be used to classify and compute (through statistical measures) optimal treatments and actions. Policy makers have set out national agendas towards such a goal; the UK's Personalised Health and Care 2020 framework set out a vision in which health and well-being data, sensed from wearable and environmental sensors, would seamlessly integrate with patient health records by 2018~\cite{Personalised2014}.  The framework proposes that these data would ``fill in the gaps'' between visits with their GP or specialist, enabling clinicians to more effectively perform personalised differential diagnosis at point of care.  It is anticipated that the introduction of such technologies will simultaneously improve outcomes substantially and drive down costs.  Other visions include use in preventative medicine, to enable early onset detection of chronic conditions, thereby potentially reversing the course of conditions before they set in~\cite{Swan2009}.  Exploratory data mining of sensed data is also likely to radically improve our understanding of both prominent and rare medical conditions, by revealing relationships among genetic, lifestyle, and environmental factors that might contribute to patients' symptoms.  Initial enthusiasm for this vision is also evident in the US, with the US Food and Drug Administration's approval of  consumer tracking devices for clinical trials, citing the importance of quantifiable analysis of physical activity to physiological monitoring~\cite{U.S.FoodandDrugAdministration2014}.

% quantified self vs self tracking vs lifelog vs prescribed sensors vs personal health records

% dont have much basis of saying this

Responding to this interest, the race to attract end-user consumers to better and more accurate physiological sensors has taken off past 8 years. New, freshly funded companies such as Withings, FitBit and Jawbone, have been joined veteran sport performance monitoring device makers such as Garmin, while established brands for sport to consumer electronics, such Nike and Sony have joined the fray to each develop attractive wearable personal tracking devices.  While still a minority, several, namely at the time of writing iHealth and Withings, have had several of their consumer-level sensors certified for clinical use, as under the US FDA Class-1 and 2 medical device classification scheme. 

%\section{Barriers to Adoption}

% Yet despite this industry, political and tech-driven enthusiasm for use of self-logged QS data, clinical uptake has been slow.  For example, even with Apple's partnership with US healthcare providers around the targeted of their HealthKit framework have focused primarily on its use in telemonitoring for randomised controlled trials (RCT) rather than for differential diagnosis.  But even here, problems have been met with some partners pulling out, citing insufficient interest by both patients and physicians~\cite{}.  

% An informal survey by sullivan_guess_2014 found that clinicians were largely ''simply not interested in FitBit data'', assigning blame on several factors~\cite{sullivan_guess_2014}. The first was time; doctors cited simply not having the time to look at data brought in by patients.  The second pertained ot access; electronic medical records simply did not admit or integrate patient-supplied data.  Third were concerns about data quality, e.g. that devices available today were simply not yet accurate enough for clinical use.  Finally, doctors expressed legal concerns around use of the data; citing the potential to be held liable for data captured by wearable health sensors.

\subsection{Evidence Based Medicine: Clinical Decision-Making with Data under Uncertainty} 

The term \emph{evidence-based medicine} refers to the use of epidemiological methods in both patient-level decision making and formulation of population-level clinical care guidelines.  In the first use of the term, David M. Eddy pointed out that, while outcome analysis may make it seem that there is a clear path towards achieving more accurate, or at least consistent, diagnoses, things are never that simple in practice~\cite{eddy1990challenge}:

\begin{quote}
  Decisions might be variable but they are not whimsical or flippant. The variability occurs because physicians must make decisions about phenomenally complex problems, under very difficult circumstances, with very little support. They are in the impossible position of not knowing the outcomes of different actions, but having to act anyway.
\end{quote}

Eddy, thus, argues that, that due to the nature of the complexities of assessing individual patients' conditions under the constraints in place, care targets and practice recommendations alone will not achieve better health outcome.  Instead, working under extreme uncertainty is a necessity and should be embraced with the acknolwedgement of the effects of various decision-making constraints and biases that are known to exist.

\subsection{Clinical Decision-Making} 

% TODO TODO 
Clinical decision-making biases have long been studied in the medical community.  
Modern texts for nursing and clinical evaluation have formalised the diagnostic process of determining a patient's disease or condition based on available evidence, including presenting symptoms as a technique known as \emph{differential diagnosis}~\cite{thompson2002clinical}.  This is usually described as some variation of the following procedure: First, the physician gathers all information available about the patient, creating a list of the patient's symptoms.  Then, she or he lists all possible candidate conditions for the symptoms, which is then prioritised with most the urgently dangerous possible causes first.  Finally, possible causes are ruled out or treated systematically.  

\subsection{Cognitive Dispositions to Respond}

Recent research has revealed that clinical decision-making is often strongly influenced by cognitive bias, leading to preventable adverse events, worsened patient outcomes, and higher mortality.  Croskerry et al. characterised a ~\cite{Croskerry2013}. \cite{Graber2002} states that cognitive bias is a major cause of preventable diagnostic error, but is extremely challenging to study and reduce. This complements earlier discoveries by \cite{Kahneman1982}, in which people tend to use biases when making decisions under uncertainty, occasionally leading to severe errors. Many forms of cognitive bias have been identified -- Wikipedia lists over 350\footnote{Wikipedia -- List of cognitive biases -- \url{http://en.wikipedia.org/wiki/List_of_cognitive_biases} [Accessed 4th Sep 2014]} -- however, the consequences of biases in clinical decision making remain largely unexplored \cite{Croskerry2013}. It is thus unknown how the provision of patient data may affect bias in clinical decisions.

% research questions
% By conducting a literature review of research in cognitive bias over the last 50 years, this dissertation presents a list of 11 cognitive biases which may affect clinical decisions made when patient data is used. From this, it has been identified that supplementing patient data may introduce further bias due to methods clinicians use to interpret patient data. Furthermore, it has been found that these biases usually have a more significant effect when decisions are made in acute scenarios, such as emergency rooms, where decisions must be made quickly with limited resources.

% definitions
%For the purpose of this paper, decisions which involve patient data are called \textit{evidence-based clinical decisions}. The term \textit{patient data} refers specifically to personal information which is collected by mobile apps and consumer devices, such as the number of steps taken, video lifelogs, location history and status updates on social networks. Patient data is subsequently \textit{interpreted} by clinicians, which refers to the observation, analysis and sense-making of data.


% stuff from pete's masters thesis

% a little bit about all the potential ways QS could help

% \cite{Swan2009} has called for use of this data as a means to supplement healthcare, which would increase ``information flow, transparency, customization, collaboration and patient choice and responsibility taking, as well as quantitative, predictive and preventive aspects.''. The use of self-tracking data in healthcare is not new. The health diary has been popular since the 1950's as a data collection method \cite{Richardson1994}.  \cite{Richardson1994} note that particular types of data that are recorded include pain, fatigue, medication use and dietary intake. This shares overlap with health and wellbeing data, suggesting there are existing use cases where health and wellbeing data could be supplemented. \cite{OLoughlin2013} has demonstrated the use of lifelogging for the purpose of increasing diet awareness. However, \cite{Swan2009} has discussed that there is currently little adoption, perhaps due to the barriers of technology in clinical environments.


\section{Pre-study: Literature Review} 

We anticipated that the slow uptake of the use of self-logged QS could potentially be attributed to a large number of factors, and thus decided it prudent to first do a survey of medical, quantified-self and HCI literature.  Our objective  was to establish a theoretical framework from which we could then identify areas that could require further investigation, as well as opportunities where research in HCI could potentially help.  

\subsection{Literature Review: Method}

We started with a set of search terms broad enough to encompass studies of clinical practice where patient-logged data (both paper-based and digital, manual and automatic) were introduced into a clinical setting.  To do this we searched PubMed, Google Scholar and the ACM DL for keywords ``patient diaries'', ``care diaries'', ``wellbeing diaries'', ``self-report diaries'', ``quantified-self'', ``self-tracking'', ``smartphone apps'', and ``wearable sensors''.

Since we wanted to focus on the usage of data by medical experts, we excluded studies about use by patients themselves, such as for feedback, reflection, goal setting and self monitoring, including behaviour-change studies and studies of motivation to self-diaries, which were prevalent from the HCI community.   Focusing only on existing practice, we omitted papers describing new interfaces and systems that have not had substantial adoption.  We also excluded papers discussing the capture side of health diarising and life-logging by patients, except where aspects of capture affected its later use.  We were careful to include papers that discussed any issues relating to the use of patient data in clinical setting, including those that discussed human factors issues specifically, to more broadly  theatres and emergency rooms.

After finding a small set of results, we then broadened our search to include studies that discussed the use of patient data in medical decision-making, including both patient-supplied and clinical data held by providers themselves.  We included ``telemonitoring'', and ``electronic patient records''.  This  hoped to find a broad range of factors spanning human-factors issues to social, cultural, institutional, situational, among others.

For each paper, we identified problem(s) in the clinical use of data, which were first added to a spreadsheet and linked to their original source.  After examining each of the papers, two researchers organised the list into themes, attempting to merge all problems with the same underlying cause, while keeping those that did not overlap distinct.  %During this process, we identified a set of themes related specifically to cognitive biases, which we analysed and report separately below.

\subsection{Literature Review: Results}

\begin{table*}[th]
    \def \w {\cellcolor{blue!40!black}\textcolor{blue!40!black}{X}}
    \setlength\extrarowheight{0.15cm}
    \setlength\tabcolsep{0.023cm}
    %\def\arraystretch{0}
    \centering
    \small
    \begin{tabular}{>{\raggedright}p{3.3cm} !{\vrule width 1.3pt}
        p{0.6cm} | p{0.6cm} | p{0.6cm} | p{0.6cm} | p{0.6cm} | p{0.6cm} | p{0.6cm} | p{0.6cm} | p{0.6cm} | p{0.6cm} | p{0.6cm} | p{0.6cm} | p{0.6cm} | p{0.6cm} !{\vrule width 1.3pt}
        p{0.6cm} | p{0.6cm} | p{0.6cm} | p{0.6cm} | p{0.6cm} | p{0.6cm} | p{0.6cm} | p{0.6cm}}
        
    %\hline
    & \multicolumn{14}{>{\centering}p{9.0cm} !{\vrule width 1.3pt}}{\textbf{Self-recorded data}}
    & \multicolumn{8}{>{\centering}p{5.1cm}}{\textbf{Patient data}} \\ \hline
    
    % use of self-recorded data by clinicians
    & \centering{\cite{chung_more_2015}}
    & \centering{\cite{ancker_invisible_2015}}
    & \centering{\cite{becker_mhealth_2014}}
    & \centering{\cite{sullivan_guess_2014}}
    & \centering{\cite{almalki_use_2015}}
    & \centering{\cite{patel_probing_2012}}
    & \centering{\cite{Swan2009}}
    & \centering{\cite{millar_shared_2004}}
    & \centering{\cite{ancker_you_2015}}
    % use of IT by clinicians
    & \centering{\cite{alsos_mobile_2012}}
    & \centering{\cite{frankel_effects_2005}}
    & \centering{\cite{zeldes_information_2011}}
    & \centering{\cite{chen_unpacking_2011}}
    & \centering{\cite{patterson_human_2004}}
    & \centering{\cite{schoenberg_weaving_2000}}
    & \centering{\cite{vikkelso_subtle_2005}}
    & \centering{\cite{hughes_2.0_2008}}
    % reliability/accuracy of self-recorded data
    & \centering{\cite{demonceau_contribution_2015}}
    & \centering{\cite{tsoukalas_data_2015}}
    & \centering{\cite{frier_hypoglycaemia_2015}}
    & \centering{\cite{khariwala_self-reported_2015}}
    & \cite{dontje_measuring_2015} \\[0.05cm] \hline
    
    \textbf{Data capture} &&&&&&&&&&&&&&&&&&&&&& \\ \hline 
    Relevance
        &\w &\w &   &   &   &\w &   &   &   &   &   &   &   &   &   &   &\w &   &\w &   &   &   \\ \hline 
    Quality
        &   &   &   &\w &   &\w &   &   &\w &\w &\w &   &   &\w &   &   &   &   &\w &   &   &\w \\ \hline 
    Completeness
        &   &\w &\w &   &   &\w &   &   &\w &   &   &\w &\w &   &   &   &   &   &\w &   &\w &   \\ \hline

    \textbf{Data access} &&&&&&&&&&&&&&&&&&&&&& \\ \hline 
    Representation
        &\w &\w &\w &   &   &   &   &   &   &   &\w &   &   &   &   &   &   &   &   &   &   &   \\ \hline 
    Selective disclosure
        &\w &\w &   &   &   &   &   &\w &   &   &   &   &\w &   &   &   &   &   &   &   &   &   \\ \hline 
    Interoperability
        &\w &\w &\w &   &   &   &   &\w &   &   &\w &   &   &   &\w &   &   &\w &   &\w &   &   \\ \hline
        %\midrule
    \textbf{Clinical practice} &&&&&&&&&&&&&&&&&&&&&& \\ \hline
    %Interpretation
    %    &   &   &   &   &   &   &   &   &   &   &   &   &\w &\w &   &   &   &   &   &   &   &   \\ \hline 
    Data literacy
        &\w &   &\w &   &\w &   &\w &   &   &   &   &   &   &   &   &   &   &   &\w &   &\w &\w \\ \hline 
    Doctor-patient relationship
        &\w &   &   &   &   &   &   &\w &\w &   &   &   &   &   &\w &\w &   &\w &\w &   &\w &   \\ \hline 
    %Professional autonomy
    %    &   &   &   &   &   &   &   &\w &   &   &   &   &   &   &   &   &   &   &   &   &\w &   \\ \hline 
    Legal issues
        &   &   &\w &\w &   &   &\w &   &   &   &   &   &   &   &   &   &   &   &   &   &\w &   \\ \hline
        %\midrule
    \textbf{Situational constraints} &&&&&&&&&&&&&&&&&&&&&& \\ \hline
    Time
        &\w &   &   &   &   &   &   &   &   &   &   &   &   &   &   &\w &\w &   &\w &   &\w &   \\ \hline
    Information overload
        &   &   &\w &\w &   &   &   &   &   &   &   &   &   &   &   &   &\w &   &   &   &   &   \\ \hline
    %Effort
    %    &\w &   &   &   &   &   &   &   &   &   &   &   &   &   &   &   &   &   &   &   &   &   \\
    
    %\hline
    \end{tabular}
    \caption{Major themes identified within the literature review, separating papers by whether they discussed self-tracked data, or patient data such as telemonitoring or EMRs.}
    \label{tbl:themes}
\end{table*}

From an initial set of 2340 results, we identified 429 papers that actually contained at least one of the search terms among keywords and the abstract. From here, we identified 22 relevant papers according to the criteria defined above.  From this we derived the 11 themes visible in Table~\ref{tbl:themes}. We categorised the these into four categories: data capture, data access, clinical practice and situational constraints. %, omitting themes pertaining to cognitive dispositions to respond which we discuss next.

\subsubsection{Data capture}

This category pertains to how individuals conduct self-tracking, and the devices used for doing so. Three themes were identified in this category:

\begin{itemize}
    \item \textit{Relevance}. Individuals have little guidance from clinicians on what they should track, so don't necessarily know what information is relevant to their condition. Without guidance, individuals may devise their own systems of recording information which are not complementary to a clinician's training. They may present irrelevant information to a clinician, or, conversely, may fail to disclose relevant information which they thought was irrelevant. Clinicians may dismiss information which is inapplicable to the current situation.
    %data may be irrelevant, or useless within the context potential barriers to adoption of self-recorded data within clinical practice, derived from literature of clinicians use of. Inapplicability IT/mHealth/telemonitoring/patient-recorded data. 
    
    \item \textit{Quality}. Many consumer devices available for self-tracking and self-logging are not clinical grade, with only a small number having been approved for medical use. Clinicians often perceive the data quality and sampling from non-clinical grade devices to be poor and perceive the devices to be unreliable. Clinicians often are unwilling to use such information out of fear of consequences. Studies have however demonstrated some devices to have good reliability, including the Fitbit wearable tracker.
    
    \item \textit{Completeness}. It has been observed that individual's who self log often do so sporadically, leaving out information which may be important. Individuals attributed this to cumbersome tools and the fact that it takes time and effort to track. It has been observed that clinicians look for detailed information to identify patterns and anomalies in data.
    %data incomplete, also, is the patient able to provide needed information (e.g. their experiences during an episode)?
\end{itemize}

\subsubsection{Data access}

This category pertains to factors contributing to how individuals access and use self-tracking data. Three themes were identified in this category:

\begin{itemize}
    \item \textit{Representation}. Health providers use standardised forms for certain types of data, including blood pressure, symptom history and glucose level. However, due to the quantity and variety of self-tracking apps and devices, there is a lack of standardisation of how self-logged data is represented. Furthermore, applying standards to data can remove important parts. Exercise logs, for example, often have distinct data structures and different meanings of `activity level'. It has been observed that this heterogeneity in data representations adds difficulty interpreting self-logged data.
    %flaggin up of things which are not noteworthy. A lack of flexibility in how information is presented .
    
    \item \textit{Selective disclosure}. When presenting information to a clinician, the patient may not realise what information is relevant and fail to disclose it. Some patients raised privacy concerns with sharing information with health providers, raising concerns that information may be used against them. Some patients therefore concealed certain information from their health providers.
    
    \item \textit{Interoperability}. A lack of standardisation of how data is represented creates problems when trying to share between colleagues and other health providers. There are concerns that self-tracking tools will be poorly integrated into existing health provider tools. In particular, there are challenges regarding the integration of Web-based services -- which many self-tracking devices and apps are -- into legacy systems. Many self-tracking apps do not provide anyway to export data, or they only do so in propriety APIs, which are not intuitive for clinicians to use.
\end{itemize}

\subsubsection{Clinical practice}

This category pertains to how clinicians' training affects how effectively self-tracking data can be used, and the potential barriers to introducing self-tracking data to clinical practice. Three themes were identified in this category:

\begin{itemize}
    %\item \textit{Interpretation} - one physician may use data in a different way to another
    
    \item \textit{Data literacy}. Clinicians are concerned that they do not have appropriate expertise or training to effectively use or validate self-logged data. Contributing factors include the large variety of forms of self-logged data, the lack of standards for data representations, the lack of access to appropriate electronic tools for analysis, and not being familiar with new tools for self-quantisation. Clinicians may rely on specialists to interpret the information, but they may not always be available.
    
    \item \textit{Doctor-patient relationship}. It has been found that mobile devices can draw attention away from the patient. User interface and physical form factor affect how doctors and patients conduct face-to-face discussion and nonverbal communication. These aspects are crucial because they allow clinicians to better understand the patient's condition, and contribute to patient satisfaction. Exam rooms are often laid out such that clinicians can quickly switch between looking at a computer and the patient, but poor user interface can cause clinicians to spend much longer looking at the computer and disengaging from discussion with the patient. Some clinicians are concerned that patients self-logging may threaten professional autonomy, with concerns that patients may begin to make decisions which would be better made with clinical advice.

    %takes time away from patient. e.g. the patient get's upset if the QS is disregarded. in modern gp practices, EMR screen is next to patient, so doctor can switch between patient and EMR quickly - incompatible with planned out environment.  Patient talks to clinician, clinician is trying to use computer so doesn't listen 
    
    %\item \texti{Professional autonomy} - "Professional territorialism and perceived threat to professional autonomy and/or scope of practice" [32] harmful to patient relationship: patient self-diagnosis
    
    \item \textit{Legal issues}. Clinicians have raised concerns that the lack of regulation on self-tracking apps and devices has implications for safety, security, data protection, and reliability of using their associated data. Health providers have also raised the issue of liability, particularly if data fell into the wrong hands and exposed them to privacy violations (HIPAA), or if decisions were made based on poor quality data.
\end{itemize}

\subsubsection{Situational constraints}

This category pertains to constraints that exist within healthcare situations. Two themes were identified in this category:

\begin{itemize}
    \item \textit{Time}. Clinicians often need to work within tight time constraints, within which they are already under pressure. Clinicians have been skeptical about how effectively they could use self-logged data within these time constraints. Some said that they could be more effective when working between visits, but that work is not recognised in those times, so are discouraged from doing so. It has been observed that computer interfaces which time a long time to use are frustrating for clinicians. It has been suggested that information overload leads to working inefficiently within time constraints. Clinicians are concerned there is also the time take to document how such data is used, even when it is irrelevant to the current situation.
    
    \item \textit{Information overload}. The quantity of information generated by self-tracking tools may be a liability, reducing a clinicians productivity and morale. The way in which information is presented contributes to the extent to which overload affects a user.
    
    %\item \textit{Effort}. perceived usefulness. belief that it won't be worth it - is this a rationale belief? not valuable
\end{itemize}

\subsection{Literature Review: Conclusion}

This literature review has confirmed that a wide variety of factors may contribute to how self-tracked data is used by clinicians. By categorising these themes, we suggest that the factors are introduced in multiple stages. First, data capture is facilitated by patients using self-tracking tools, and the reliability of the tool and how the patient uses it affects how useful a clinician will find that information. Second, patients provide the information to the clinician, perhaps mediated by the self-tracking tool, and clinicians attempt to access and use the data. Third, the clinicians must use this information according to their training and practice. Fourth, the clinician must interpret the information within tight time constraints.

%introduce Markov model?

% We anticipated that the slow uptake of the use of self-logged QS could potentially be attributed to a large number of factors, and thus decided it prudent to first do a survey of medical, quantified-self and HCI literature.  Our objective  was to establish a theoretical framework from which we could then identify areas that could require further investigation, as well as opportunities where research in HCI could potentially help.  

%% party here

% \subsubsection{Cognitive Dispositions to Respond}

% During the first phase, we encountered a substantial body of work pertaining to  situational and psychological factors that influenced the decision-making performance of medical experts in clinical settings.  Croskerry's \emph{Cognitive Dispositions to Respond}~\cite{} 

% We started with MEDLINE, Web of Science, and the ACM DL
% Using the keywords "medical decision-making", "clinical decision-making", "patient-logged data", "quantified self", "data in hospital admission", "GP visit", searches were performed on BMJ, MEDLINE, CINAHL, Google Scholar, Web of Science, Springer, and the ACM DL.  This collectively resulted in 2982 results, many of which
% d
% Temporal constraints
% Situational constraints
% UI/UX constraints 
% Workflow constraints
% Cognitive Biases
% Training/background constraints
% Exclusion criteria:
%   Anything that only involved the patient, including behaviour change studies


%   UK Prime Minister Gordon Brown,  pronounced in 2008 that more effective ``future will be one of patient power, patients engaged and taking control over their own health and healthcare.'' – Gordon Brown, U.K. Prime Minister


\section{Methodology}
\subsection{Recruitment}
We recruited 10 participants (4 female) who were full time, highly experienced doctors.  Three participants were general practitioners in England, practicing in three different urban settings; seven were board-certified specialists from a variety of specialisms such as rheumatology, nephrology, cardiology, hepatology, practicing at a single large urban hospital in the United States.  We used a small snowball technique to build our population of participants.   We recruited both specialist hospital doctors and general practitioners with the expectation that this would provide a variety of distinct approaches to selecting, judging, and using information and data.

\subsection{Data Collection}
We conducted semi-structured interviews using a protocol designed around the following areas of interest:
\begin{itemize}
\item How would doctors judge patient-supplied data?
\item Would doctors use patient-supplied data?
\item How does patient-supplied data align with current work flows and work practices?
\end{itemize}
We drafted two written scenarios of patients presenting with a set of symptoms.  These were actual scenarios of real patients selected from set of scenarios derived from the “Think Like a Doctor” column in the New York Times.  We edited them so that they would be understandable to doctors in the US and UK, and added further information on the the nature of patient-provided data that did not obscure a path toward a diagnosis or care plan.  The scenarios were written as narratives that outline the patient’s symptoms, some background and history, and the reasons why the patient decided to see the doctor.  

In the first scenario, a middle aged man finds it difficult to sleep, feels nauseous, sweaty, out of breath, and finds, when lying in bed, that his legs disobey his mental orders to stop moving.   There is some background history including that he has a patent foramen ovale (PFO, or a small hole in the tissue separating the right and left atria), suffered a mild stroke years ago, is on blood thinners, experiences occasional panic attacks, and worked in the steel and automotive industries.  We added to the scenario that his wife convinced him to become a strict vegetarian, and she bought him a heart monitor that he regularly uses to check his pulse. When he visits the doctor, he brings along a printed Excel chart that shows his resting heart rates taken mornings, afternoons, and late evenings for the past month. Aside from a few spikes of 130 beats per minute, the heart rates hover around 85.  His pulse did, however, increases to about 100 the three days before his doctor visit.  The New York Times scenario does not include this patient’s vegetarian diet, nor does it include that he has a heart monitor that he uses daily—this is information we added to the scenario.

In the second scenario, a university student is lightheaded, dizzy, her body feels heavy, feels anxious, has blurred vision, has headaches, and she is prone to fainting if she tries to stand. She had surgery on her back twice about 18 months ago: once to fix to fix a pinched nerve, and again to cut out an infection that rooted there.  She had been on antibiotics after the surgery, and the infection had cleared.  She does not smoke, and her only alcohol consumption was an occasional glass of wine.  She drinks copious amounts of coffee to keep her awake, especially during exam periods, and she was worried about this.  She installed an app on her iPhone that kept a record of her caffeine intake, and she this used whenever she would drink coffee. She kept a log that she printed and brought with her when she visited her doctor. The log chart shows that she was consuming in excess of 1000mg of caffeine a day, until the day before she visited her doctor when she had no coffee.  The New York Times scenario does not include this patient’s caffeine intake, nor that she measures it.  This is information we added.

None of the participants had seen these scenarios in advance, and none knew in advance the true diagnoses—the man has a vitamin B12 deficiency, and the student was diagnosed with postural tachycardia syndrome.  We chose these scenarios because they are relatively complicated, such patients would be seen by both GPs and specialists, a diagnosis is not obvious, and they would provoke a chain of thoughts and judgments by the participants.

We distributed a scenario and a chart to the participants, and asked them to think out loud as they were reading them.  We then asked for general comments on the case, and how they would use the information supplied by the patient.  General practitioners were interviewed individually; specialists read and commented on the scenarios one-by-one in a group.  In addition, we asked specialists to jot down their thoughts on printed scenarios, and to hand them in at the end of the session.  Interviews lasted for about an hour, and they were recorded and transcribed.

\subsection{Data Analysis}
The research team individually analyzed each transcript, swapped analyzed transcripts, and re-analyzed them. \cite{pope2000qualitative}  In the first iteration, we developed a set of keywords (such as “Training” “Patient History” “Exam”) that characterized what the participants were saying as they read through the scenarios and then discussed them, as well as what they had written on the printed scenarios that we distributed. At this stage, we were interested in only in identifying and tagging what they said—this was the qualitative equivalent of performing simple descriptive statistics on a dataset.  Using inductive qualitative methods, \cite{corbin2014basics} we iteratively developed a coding scheme related to participants’ information and data acquisition practices, their judging and trusting data and information, the patterns of interrogating data and information. Subsequent analyses allowed us to discern how data and information content and structure support or hinder their work practice and flow of thoughts.  Our final analysis helped us focus on the cyclic nature of their strategies, and helped us construct a model of this.


\section{Results}

Participants in the focus group - seven specialists in the US (six male) and three NHS GPs in the UK.

    
    
\subsection{Reorganising information into a medical history}
Five hospital specialists (S1, S2, S3, S4, S5) stated that they needed to reorganise the information that was given to them in the form of a medical case. In particular, they wanted a more complete history so they could understand how the patient’s current even fits within a timeline, or whether it’s a distinct event. GP1 said he needed more specific information about his medication, rather than just ``a couple of blood thinners''.

\begin{quote}
    ``for me, there's an incomplete history here. So, you know, the first place I would go is back to the patient, and try to pull out, what are the details of the timeline really, the timeline. Is this a distinct event or are this linked somehow, and pull it out from her, the history.'' S2
\end{quote}

Particularly prominent within the session with hospital doctors was the need to reorganise the scenario that aligns and supports their work routines. One hospital specialist explained that it was difficult for them to follow the medical case in the scenarios because they did not provide the patient’s structured medical record which would give greater detail into the patient’s history and flow of interventions in a standard format.

\begin{quote}
    ``this has been going on for a longer period of time and then I got the sense that it started 18 months ago but you know, I would like the confirm that. [...] Was that when it really started? Or is there, in this pattern, you know, along the lines of is this real pathology is this been this pattern of it going on and frequent doctor visits and nothing found, uhm, you know it’s just anything more has she had recent blood work has she had recent x-rays or tests that might tell us that there's real pathology or that there's something to focus in on uhm you know, those kinds of things.'' S3
\end{quote}

Because there was no patient, some clinicians role-played the patient:

\begin{quote}
    ``She's a student, third year… probably coming up to exams. Is it a tension headache?'' GP1
\end{quote}

\begin{quote}
    ``She's a student, I suppose I'm already thinking alcohol psychology.'' GP2
\end{quote}

\subsection{Evaluating the data quality and completeness}

Concerning the heart rate chart, GP1 stated that he would need to understand how the patient recorded that data. One important factor raised was the need to understand if any illness or medication had interacted which may contribute to a change in the patient’s heart rate. GP2 and S2 also mentioned he would need to understand what the patient was doing at the recorded times, particularly during the high peaks. In addition, S2 needed to know what symptoms the patient was experiencing at those times. 

The heart rate chart also provoked comments on the reliability of the data, relating to both the accuracy of the recording equipment and how the patient recorded the data. GP1 explained that he could not assume that the data was ``objective'' enough, and that spot checking wasn’t enough - he would instead need a trace. He felt he would need to take measurements using his own calibrated equipment, which he could trust:

\begin{quote}
    ``I want to use my machine, which has been precalibrated, not off the shelf, because I don't know about this machine's calibration. Can I trust all the data? No. Can I assume all the data is correct? No. So I have to use a calibrated machine which I feel happy with, that I've used before, and which has an electrician or someone who said this machine is pukka. And then I apply that test; a very quick non-invasive test. I can do a heart trace in here [the GP’s office] to see if I can spot anything as well'' GP1
\end{quote}

Conversely, S2 suggested that home-monitoring may be more accurate than office-based monitoring due to white-coat hypertension, and the fact that home-based monitoring may happen over a longer period of time.
Understanding the patient’s motivations
The patient’s motivations for recording the data was also a factor in determining whether the data should be used. Pertaining the caffeine chart, GP3 stated a need to understand why the patient recorded it, when it is not a normal thing to do:

\begin{quote}
    ``I would try and ask a little bit more about this [referring to caffeine chart] and why she's done this anyway, just to have an understanding, you know, of reasons, because not everyone charts their caffeine intake.'' GP3
\end{quote}

In particular, S3 suggested that the presentation of the caffeine chart was a way of indirectly telling him that there are underlying psychological issues, such as being overwhelmed and struggling with their studies or job. Referring to both scenarios, S5 asserted that the patient was ‘obsessive’ to record both the data, suggesting an underlying psychological issue. The mere existence of the heart rate plot provoked a similar jump to psychological issues with S4:

\begin{quote}
    ``They’re faking it. If someone brought this chart to me, there’s a red flag that this guy’s got psych issues'' S4
\end{quote}

Although S3 raised that it’s common for engineers to bring in these kinds of plots, and S7 proposed that patients bringing in data will become normal practice, the need to understand why the patient recorded the data remained a common theme throughout the discussions. S6 said that a ``lot of patients come with a diagnosis that they put on'', with S1 following on from this by saying:

\begin{quote}
    ``It's typical that patients like this that come in and they give you, you get this whole story, and then they want you to focus on this it and it takes your attention away, or they're going to tell you this is the reason why all of this is going on, and then you have to say well ok but let's just put that there.'' S1
\end{quote}

GP3 also needed to understand why both patients had brought the charts, expressing confusion at their presentation. He wanted to understand what these patients wanted out of their consultations.

\begin{quote}
    ``if all that are normal, and she's still feeling, you know, then again, I would then refer back to that [the chart], especially if she's quite... a student and knowledgeable about that, might be cause to listen. "Hang on, OK, tell me more about this, and why? You know, why are you doing that?".'' GP3
\end{quote}

GP1 explained that there is a lot of pressure for GPs to prescribe, and that it was necessary to resist this, because patients push for prescriptions of ``poorly evidence based things''. 

\subsection{Deciding how to use the patient-provided data}

In deciding how the patient-provided data should be used or not, S1 explained that they needed to initially make a decision on how long to spend looking at it

\begin{quote}
``I mean I think when you look at something like this you do have to be able in your own mind to say,’ OK am I going to spend 30 seconds on this, or am I really going to spend more time and give it more importance?’Aand I think that's what we face a lot.'' S1
\end{quote}

S1 explained that these information sources will add layers of data assessment to his practice, and questioned whether this patient-provided information may adversely affect efficient work flow

\begin{quote}
    ``The layers of information, data assessment… You know, it's ramping up and and  all of these devices are certainly adding or could and will add yet more of this, you know, data that, you know, you know…  Certainly here we get a lot of things faxed to us. We get to know patients who come here and then they go somewhere else.  But then we are getting their lab work over and over again—their X-rays, their visits. And you know at some point you have to ask yourself, you know, what is efficient here and what is not?'' S1
\end{quote}

S7 concurred, explaining that this adds ``complexity to an already complex medical interface.'' When asked if patient-provided information is a good or a bad thing, there was general agreement within the room. S4 suggested that age of the clinician may be a large factor in acceptance of patient provided data arguing that medics are ``creatures of habit'', who do not usually change their practice significantly if their current work practice satisfies their demands (???)   Newly-trained young doctors may become used to patients bringing their own data. He foresaw not changing his own practice within the next 10 years.

\begin{quote}
    ``Whereas the younger doctors that are coming in are seeing patients for the first time and they're used to people bringing in this kind of stuff and will put more thought into it rather than... Doctors are typically creatures of habit. You've been doing something for ten, fifteen years the same way, you're going to carry on doing it. There are minor changes that happen in terms of pathology and diagnostics, but in general you're used to doing it in a certain way... So I think it remains to be seen, but, you know, when I think what I'll be doing in 10 years time in terms of how I'll be managing a patient, it will be very similar.'' S4
\end{quote}

Need for physical exam, scenario and data alone was not enough. e.g. for ruling out neurological causes.

\begin{quote}
    ``I haven't even touched him yet. and then I'd do after my history, then I'd do an examination, again feeling the pulse myself, feeling is it strained, quality, character, whether I can do anything to the pulse, and then listening to his heart and doing the various movements and motions sit forward sit back, to see if he's got any murmurs associated with this. see if he's been compliant with his anticoagulants, maybe he is maybe he isn't.'' GP1
\end{quote}


\subsection{Using (or ignoring) the patient-provided data}

There were stark differences in how useful the clinicians found the patient-provided information. GP3 chose to ignore the caffeine, he stated that it was irrelevant, but conceded that he may be wrong and may come back to it later. Despite stating that he would also ignore the heart rate plot and it would not have much influence on his decisions, he did observe that there were high peaks, which motivated questions about what the patient was doing at those times. 

\begin{quote}
    ``But certainly, you know, I don't think this [chart] would influence, but I would, you know, the only influence it would have is to try and understand more about why he did this and what he wants out of the consultation'' GP3
\end{quote}

S4 said the heart rate was ``underwhelming,'' essentially showing normal heart rate other than twice. GP2 said that the information conveyed in the heart rate plot was not significant, with high rates explained by any activity he could have been doing at the time. GP2 said that blood pressure would have been more useful to her.

S4 stated that the presentation of this information would sway him toward caffeine as a cause:

\begin{quote}
    ``I think if you see her at the office and she brings you this immediately you start thinking you know, this is all from too much caffeine.'' S4
\end{quote}

GP1 said the caffeine chart suggested a coffee-withdrawal headache. He verbalised a unit conversion in trying to understand how much caffeine 400mg is.

\begin{quote}
    ``right! this could be a coffee headache. Well if you stop drinking coffee you get a headache, if you start drinking coffee you get a headache. Daily consumption -- WOW! - above 400mg - 150mg per cup. Yeah, so this could be a coffee withdrawal headache.'' GP1
\end{quote}

GP1 went on to suggest that he might advise the patient to have a graded reduction in caffeine. Note in the quote above that GP1 was able to make a unit conversion of caffeine to cups of coffee, and was also able to note that this is an excessive amount. S2, S3 and S5 said they did not know what a normal caffeine level is. Similarly, although S2 made an initial observation of the heart rate chart based on his own beliefs of how heart rate varies, he said he would need to consult a cardiologist to fully understand it:

\begin{quote}
    ``Well one thing that struck me is how little variability there was in the heart rate during the time of the day. I thought that was, I would need to ask a cardiologist, but I thought there was greater variability in heart rate.'' S2
\end{quote}


\subsection{Pattern recognition, realising it might not be anything}

\begin{quote}
    ``because I don’t know what it is, the other things is - it’s funny on the charts - the last three days are kinda an uptrend. so maybe go talk to the patient and ask: why now, for the last three or four days, are they slowly going up? Whereas before it was up and down. It might not be anything.'' S6
\end{quote}

One hospital specialist said he would elicit this information directly from the patient. This need for the patient in the room was reciprocated by one GP, who explained that they observe how the patient behaves - how they walk in the room, how they sit down. 

\begin{quote}
    ``the way they sat, or something else, or stomping gate, or something. you're eyes just think oh! and then your connection will be made consciously and unconsciously'' GP1
\end{quote}


\subsection{Using patient-provided data to support risk averse workflow}

S1 made connections between the cases, GP2 used prior experience

Both S3 and S4 explained that they reflect on the scenarios based on their own experience and training within their specialisms

\begin{quote}
    ``as a pulmonologist, which I am, you know, I noted the history of asthma, and I said whatever's going on is not exacerbating her asthma'' S3
\end{quote}


\subsection{Linking, unifying}

\begin{quote}
    ``we as physicians - how we're trained - I can't help but try to figure out a way to lump all this together, to come with one unifying diagnosis.'' S2
\end{quote}

\begin{quote}
    ``what are the details of the timeline really, the timeline. Is this a distinct event or are this linked somehow, and pull it out from her, the history.''  S2
\end{quote}

S5 also looked for the worst possible thing

\begin{quote}
    ``really just looking for things that are more serious, like trying to see what's the worst possible thing that she could have, approaching it that way. And hopefully it will turn out to be nothing. but I guess that’s, usually, I'm always looking for what's the worst possible thing the person could have and work backwards from there.'' S5
\end{quote}

\begin{quote}
    ``trying to approach it like we would in an office, really just looking for things that are more serious, like trying to see what's the worst possible thing that she could have, approaching it that way'' S5
\end{quote}

GP1 checks with other GPs. GP1 - ``common things are common, common presentations of rare conditions are less common than rare presentations of common conditions.''. GP1 said that the knowledge of caffeine consumption changed his perspective on possible causes, though said that although caffeine may seem likely, there could be a far more serious tumor. On the other hand, you have to be economic. A plan which is both economic and risk averse

\begin{quote}
    ``but at the moment I've chopped chopped chopped chop, and we come to here. And now I think right, we've pruned off all of that, now I've got the bare tree - it's distiguous, not perennial. and it's very easy to see, this is my path now. it's your heart, mate. and I need to do just one or two tests to show otherwise this tree... this root or trunk, becomes thicker and I will go that way. that's how I think.'' GP1
\end{quote}


\subsection{Summary of findings}

\section{Discussion}

% = Intro - summarise findings, introduce themes=


% == Summarise relevant findings ==
%    lack of integration into workflow

% == Relate to literature ==

Differences in how patient data is used
Standardise it?

\subsection{Design for Constraints} % (support & respect clinical workflow?)

% What did we observe?
% EXPERTISE, SPECIALITY, COMPLEXITY, TIME, EFFORT 
% How do they relate to the literature. did it correspond to a particular theme? support or contrast?
% DATA LITERACY, SITUATIONAL CONSTRAINTS[TIME,OVERLOAD,EFFORT]
% How can we make QS apps better?


Our literature review revealed that clinicians are concerned with their lack of expertise in using self-logged data and that it may be inefficient to use within tight time constraints. Within our experiment, we observed that clinicians did indeed not always have the expertise the effectively use the patient-provided data. This was particularly prevalent with the caffeine chart, with many of the clinicians unable to effectively use the information because they weren't familiar with a normal caffeine level. Similarly with the heart rate chart, a number of clinicians said they were not confident in interpreting the data, and would need consult a cardiologist to help interpret or validate their findings. 

Building on this, clinicians raised concerns that the introduction of patient-provided data may add complexity to their practice, and they would need to make a judgement on whether using this information is efficient, and how much time they should spend on it. Supporting this judgement may therefore allow self-logged data to be more effectively used within clinical settings -- by visually illustrating the relevance of self-logged data to a scenario, the quality of the data, whether it compliments the clinician's training, and so on, a clinician can more quickly and reliably decide whether to use it.


\subsection{Support risk aversion} % (support & respect clinical workflow?)


% What did we observe?
Our findings suggest the clinicians followed different workflows. Number of GP's small. Big variation in style of practice. Risk mitigation in common. 
Suggest that Quantified Self data, when introduced, should be able to support risk mitigation - that is, a clinician rules out the worst diagnoses first.
Reflecting on their training:
* RISK AVERSE: Clinicians followed a risk averse strategy - rule out worst first (Ranking hypotheses by risk; what if it's a tumor?)
* STATISTICAL: Clinicians based their judgements on statistics (Common things are common, but not helpful in rare conditions)
* IRRELEVANT: Clinicians discarded data because it was not relevant to the worst case
* WITHHOLD JUDGEMENT: Clinicians were cautious to intervene (Withholding judgement; keep them alive without interfering too much)
* UNIFYING DIAGNOSIS: Clinicians tried to link information together to find a unifying diagnosis

% How do they relate to the literature. did it correspond to a particular theme? support or contrast?

* RELEVANCE

% How can we make QS apps better?




\subsection{Need to understand patient motivations}

% What did we observe?

% How do they relate to the literature. did it correspond to a particular theme? support or contrast?

* SELECTIVE DISCLOSURE
* PATIENT SATISFACTION

% How can we make QS apps better?


Clinicians should be aware that more people will do this (they're not obsessed)
Patients will be bringing more data
QS should highlight dangers of self-diagnosis (patients should not be coming with a diagnosis)
Data as a symptom
* GP said he would respect patient's efforts
* will patients be dissatisfied when doctors dismiss data

\subsection{QS is a supplement - patient needs to be there}

% What did we observe?

% How do they relate to the literature. did it correspond to a particular theme? support or contrast?

* COMPLETENESS

% How can we make QS apps better?


need for physical exam, how they walk in, talking to patient
implications for doctors on social nets
stereotyping?
* doctors ask about what the patient was doing at the time of recording
Immediate dismissal of data
Psych issues.

\subsection{Representation: presentation in form of a medical record} % (representational flexibility)

% What did we observe?

% How do they relate to the literature. did it correspond to a particular theme? support or contrast?

* QUALITY
* COMPLETENESS
* REPRESENTATION
* INTEROPERABILITY
* INTERPRETATION
* DATA LITERACY [already mentioned]

% How can we make QS apps better?

* Confirmation bias
* Interpreting according to personal belief
* Data more trustworthy when it appears normal
* structured in a way clinicians are used to
  * Training / habit / changes happen slowely / won't change
  * adopts standard that have been designed in safety critical scenarios
    * pattern recognition
    * bias (illusory patterns)
    * standardisation of units
    * standardisation of design practices
* timeline / linking with other information
  * placed in context of patient history - what else was happening on those dates?
  * comparing data to previous days
  * comparison to normal (what's normal for this patient?)
  * highlight inconsistencies / contradictions
* however, clinicians didn't always trust the data
  * clinicians needed validation, want to retake readings
  * devices need to be regulated
  * need for provinance?

\subsection{Limitations}

Number of participants
How scenario was presented (not a medical record, no patient to examine)
Low granularity information, printed on paper

\subsection{Future Work}

it's not just what the data is, it's who it comes from {your friend, your colleague, your doctor}. is it {a teenager, a female student, a male student, a middle aged man, a middle aged woman}


\section{Conclusions}


% REFERENCES FORMAT
% References must be the same font size as other body text.
\bibliographystyle{SIGCHI-Reference-Format}
\bibliography{wsi-chi-16}

\end{document}

%%% Local Variables:
%%% mode: latex
%%% TeX-master: t
%%% End:
